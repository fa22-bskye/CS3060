\documentclass[paper=letter, fontsize=11pt]{scrartcl} % A4 paper and 11pt font size

\usepackage{enumitem}
\usepackage{listings,multicol}
\usepackage[T1]{fontenc} % Use 8-bit encoding that has 256 glyphs
\usepackage{fourier} % Use the Adobe Utopia font for the document - comment this line to return to the LaTeX default
\usepackage[english]{babel} % English language/hyphenation
\usepackage{amsmath,amsfonts,amsthm} % Math packages
\usepackage{lipsum} % Used for inserting dummy 'Lorem ipsum' text into the template
\usepackage{sectsty} % Allows customizing section commands
\allsectionsfont{\centering \normalfont\scshape} % Make all sections centered, the default font and small caps
\usepackage{fancyhdr} % Custom headers and footers
\pagestyle{fancyplain} % Makes all pages in the document conform to the custom headers and footers
\fancyhead{} % No page header - if you want one, create it in the same way as the footers below
\fancyfoot[L]{} % Empty left footer
\fancyfoot[C]{} % Empty center footer
% \fancyfoot[R]{\thepage} % Page numbering for right footer
\renewcommand{\headrulewidth}{0pt} % Remove header underlines
\renewcommand{\footrulewidth}{0pt} % Remove footer underlines
\setlength{\headheight}{13.6pt} % Customize the height of the header

\setlength\parindent{0pt} % Removes all indentation from paragraphs - comment this line for an assignment with lots of text

\usepackage[margin=0.75in]{geometry}
\usepackage{hyperref}
%----------------------------------------------------------------------------------------
%   TITLE SECTION
%----------------------------------------------------------------------------------------

\newcommand{\horrule}[1]{\rule{\linewidth}{#1}} % Create horizontal rule command with 1 argument of height

\title{ 
    \normalfont \normalsize 
    \textsc{CS 3060 Programming Languages, fall 2022} \\ [25pt] % Your university, school and/or department name(s)
    \horrule{0.5pt} \\[0.4cm] % Thin top horizontal rule
    \huge Assignment \#3 \\ % The assignment title
    \horrule{2pt} \\[0.5cm] % Thick bottom horizontal rule
}

% \author{John Smith} % Your name

% \date{\normalsize\today} % Today's date or a custom date

\begin{document}

    \begin{center}
         Assignment \#3\\
        \small CS 3060 Programming Languages, Fall 2022 \\
        \small Instructor: S. Roy \\
        \huge Scala \#1
    \end{center}
    
    \textbf{Due Date:} Problems of "hw3a" are due on Oct 15 whereas others on Oct 21. \\
    \textbf{Total Points:} 60 points \\


    \textbf{Directions:} Using the source provided via Gitlab \@ \texttt{https://gitlab.com/sanroy/fa22-cs3060-hw/},
complete the assignment below. The process for completing this assignment should be as follows:

    \begin{enumerate}[noitemsep]
        \item You already forked the Repository ``sanroy/fa22-cs3060-hw'' to a repository ``yourId/fa22-cs3060-hw'' under your username. If not, do it now.
        \item Get a copy of hw3 folder in ``sanroy/fa22-cs3060-hw'' repository as a hw3 folder in your repository ``yourId/fa22-cs3060-hw''
        \item Complete the assignment, committing changes to git. Each task code should be in a separate file. As an example, task6.scala for Task 6.
        \item Push all commits to your Gitlab repository
        \item If you have done yet done so, add TA Akhil (username: ayerrab) and Roy (username: sanroy) as a developer of your Gitlab repository
    \end{enumerate}


    \textbf{Tasks:}

      \begin{enumerate}

        \item \textbf{(4 Points. part of hw 3a) Task \#1:} Write a Scala program 
which asks the user to type 3 lines (e.g., before going to the
next line the user will hit the 'Enter' key, etc.) on keyboard, 
and saves the lines to a file named "file.txt". Then, the program opens the same file and counts the number of words 
and reports the number. Your program file should be named as task1.scala and should have necessary documentation. 
Also, create a README file, showing one sample running of your program and the output.

        \item \textbf{(4 Points. part of hw 3a) Task \#2:} Write a Scala program which asks 
the user to type the name of a file.
If the file-content (Note: we are NOT talking about the filename string)
contains ``cpp'' or ``ruby'', then print ``The file content is good''. 
If the file-content contains ``haskell'' or ``scala'',
then print ``The file is awesome''. Otherwise, print "The file is boring".
Your program file should be named as task2.scala and should have necessary documentation. 
Also, create a README file, showing one sample running of your program and the output.

        \item \textbf{(4 Points. part of hw 3a) Task \#3:} Write a Scala program which 
prints the string ``The square root of $x$ is $y$'' 20 times 
while substituting $x$ by numbers from 5 to 24 where $y$ is $x^{1/2}$.
Your program file should be named as task3.scala and should have necessary documentation. 
Also, create a README file, showing one sample running of your program and the output.

\item \textbf{(7 Points. part of hw 3a) Task \#4:} Write a Scala program called $sumOfPower$ to calculate the 
sum $1^1 + 2^2 + 3^3 + ... + 10^{10}$ without using an exponent operator.  
You can do this using nested \emph{for} loops.  Verify:  The sum equals 10405071317.
Your program file should be named as task4.scala and should have necessary documentation. 
Also, create a README file, showing one sample running of your program and the output.

\item \textbf{(7 Points. part of hw 3a) Task \#5:} Write a function called $doSplit$ which, 
given a string and a specific character, 
return a list which is substrings of the original string 
from one instance of the specific character to the next.  
Of course, do this without using built-in functions to the extent possible.

An example: if the given string is pq\$xyz\$\$ab\$c and given char is \$, 
then the output should be List("xyz", "", "ab"). 

Your program file should be named as task5.scala and should have necessary documentation. 
Also, create a README file, showing one sample running of your program and the output.

\item \textbf{(8 Points.) Task \#6:} Write a Scala program to find 
if a given number is a Kaprekar number. 
9 is a Kaprekar number since
$9^2 = 81$ and $8 + 1 = 9$

297 is also Kaprekar number since
$297^2 = 88209$ and $88 + 209 = 297$.

In short, for a Kaprekar number k with n-digits, 
if you square it and add the right n digits to the left n or n-1 digits, the resultant sum is k. 

Your program file should be named as task6.scala and should have necessary documentation.
Also, create a README file, showing one sample running of your program and the output.

 \item \textbf{(14 Points) Task \#7:} Go to \texttt{http://www.textfiles.com/stories/} and check that this site
\footnote {Disclaimer: we did not really check whether this website contains any improper story or language.
If you find something improper, please ignore this site and use some other source} hosts multiple stories
while each story is in a textfile. Download two textfiles of your choice, which have atleast 600 words,
and save the files as \texttt{story1.txt} and \texttt{story2.txt}.
Your program needs to read these files and process
them to collect some statistics. In particular, for each story $x$ report the total number of unique words 
(i.e., without counting repetition) in $x$. Also, report
the third-most frequent word in $x$ and its frequency. 
Also, find the number of common (and unique) words over these two stories
(i.e., if both the stories have a same word $w$, then we consider that there is one common word $w$).
\textbf {Hints:} You may use List, Map (or HashMap), and Set data structures as they are available in Scala.
You may design a regular expression to define a \emph{word}.
Writing README file carries 2 points.

 \item \textbf{(12 Points) Task \#8:}  Write a function foo that takes two lists of integers and returns a  
list of tuples as explained with the following examples. You are not allowed to use any library function.

foo(List(1,2,3), List(21, 22, 23)) returns List((1,21), (2,22), (3, 33))

foo(List(1,2,3), List(21, 23)) returns List((1,21), (2,23))

foo(List(1,2), List(21, 22, 23)) returns List((1,21), (2,22))

Note that if the lists have unequal number of items, then foo ignores the extra items in one of the lists.
Writing README file carries 2 points.
    % \vspace{2cm}

    \end{enumerate}

\end{document}


