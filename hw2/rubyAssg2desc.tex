\documentclass[paper=letter, fontsize=11pt]{scrartcl} % A4 paper and 11pt font size

\usepackage{enumitem}
\usepackage{listings,multicol}
\usepackage[T1]{fontenc} % Use 8-bit encoding that has 256 glyphs
\usepackage{fourier} % Use the Adobe Utopia font for the document - comment this line to return to the LaTeX default
\usepackage[english]{babel} % English language/hyphenation
\usepackage{amsmath,amsfonts,amsthm} % Math packages
\usepackage{lipsum} % Used for inserting dummy 'Lorem ipsum' text into the template
\usepackage{sectsty} % Allows customizing section commands
\allsectionsfont{\centering \normalfont\scshape} % Make all sections centered, the default font and small caps
\usepackage{fancyhdr} % Custom headers and footers
\pagestyle{fancyplain} % Makes all pages in the document conform to the custom headers and footers
\fancyhead{} % No page header - if you want one, create it in the same way as the footers below
\fancyfoot[L]{} % Empty left footer
\fancyfoot[C]{} % Empty center footer
% \fancyfoot[R]{\thepage} % Page numbering for right footer
\renewcommand{\headrulewidth}{0pt} % Remove header underlines
\renewcommand{\footrulewidth}{0pt} % Remove footer underlines
\setlength{\headheight}{13.6pt} % Customize the height of the header

\setlength\parindent{0pt} % Removes all indentation from paragraphs - comment this line for an assignment with lots of text

\usepackage[margin=0.75in]{geometry}
\usepackage{hyperref}
%----------------------------------------------------------------------------------------
%   TITLE SECTION
%----------------------------------------------------------------------------------------

\newcommand{\horrule}[1]{\rule{\linewidth}{#1}} % Create horizontal rule command with 1 argument of height

\title{ 
    \normalfont \normalsize 
    \textsc{CS 3060 Programming Languages, Fall 2022} \\ [25pt] % Your university, school and/or department name(s)
    \horrule{0.5pt} \\[0.4cm] % Thin top horizontal rule
    \huge Assignment \#2  \\ % The assignment title
    \horrule{2pt} \\[0.5cm] % Thick bottom horizontal rule
}

% \author{John Smith} % Your name

% \date{\normalsize\today} % Today's date or a custom date

\begin{document}

    \begin{center}
         Assignment \#2\\
        \small CS 3060 Programming Languages, Fall 2022 \\
        \small Instructor: S. Roy \\
        \huge Ruby \#2
    \end{center}
    
    \textbf{Due Date:}  Sep 16. 11:59 pm\\

    \textbf{Total points:} 60 points \\

    \textbf{Directions:} Using the source provided via Gitlab \@ \texttt{https://gitlab.com/sanroy/fa22-cs3060-hw/}, 
complete the assignment below. The process for completing this assignment should be as follows:

    \begin{enumerate}[noitemsep]
        \item You already forked the Repository ``sanroy/fa22-cs3060-hw'' to a repository ``yourId/fa22-cs3060-hw'' under your username. If not, do it now. Make sure that your repository is ``private''; if you fail to do so, you will lose 20\% of the points.
        \item Get a copy of hw2 folder in ``sanroy/fa22-cs3060-hw'' repository as a hw2 folder in your repository ``yourId/fa22-cs3060-hw'' 
        \item Complete the assignment, committing changes to git. Each task code should be in a separate ruby file. As an example, task1.rb for Task 1. 
        \item Push all commits to your Gitlab repository
        \item If you have done yet done so, \textbf{add} TA Akhil (gitlab id \textbf{ayerrab}) and Roy (gitlab id \textbf{sanroy}) as a member (in `Developer' mode) of your Gitlab repository
    \end{enumerate}

    \textbf{Tasks:}
    \begin{enumerate}[noitemsep]
        \item \textbf{(12 points) Task \#1:} Write a program that takes 
a filename $f$ as input, and checks whether file $f$ contains any word 
which starts with a capital letter (i.e., A-Z) and ends with a small letter (i.e., a-z) 
and contains a vowel (i.e., a,e,i,o, or u). 
As an example, "Apple", "Pen2Pen", or so can be such a word.
Your program needs to print each line (in file $f$) that contains a word 
conforming the above pattern. 
Your program should also print the line numbers of the matching lines.
\emph{Writing readme carries 1 point.}

\textbf{Example run}: ruby $~~~$  task1.rb $~~~$ file-to-search-in $~~~$ 

An example output is as follows.

\emph{Line 3: sdf \textbf{Apple} bjAbc bjb}

\emph{Line 7: Bgjh 12bask 13hj \textbf{Pen2Pen}} 


Note: 1. Your project repository needs to include a sample file-to-search-in. 2. Do NOT use any library's grep function. You implement it on your own, which may take only few lines of code. You can utilize \emph{regular expression}.
      
  \item \textbf{(12 points) Task \#2:} Write a function that takes an array A of integers 
as the input 
and does the following: (a) Uses \emph{each} method to print the last digit of each integer in A, 
(b) Uses \emph{select} method to find all the integers (in A) whose lengths are less then 4, 
(c) Uses \emph{map} method to build a new array with the length of the integers of A, 
and (d) Uses \emph{inject} method to find the sum of the length of all integers of A.
To test the function, build an array A of 12 random integers, 
and pass A into the function as a parameter.
\emph{Writing readme carries 1 point.}
        \item \textbf{(10 points) Task \#3:}. Function3A calculates the Fibonacci Series ($F_n = F_{n-1} + F_{n-2}$) iteratively. 
Calculate all Fibonacci numbers ($F_n$) for $n=1$ to 38. 
Function3B calculates the same Fibonacci Series but using the recursion technique. 
After implementing the above two functions, compare the computation time (for doing the above calculation) 
using Ruby's benchmark library. \emph{Writing readme carries 1 point.}
\textbf{Hint}: if necessary, Ruby lecture slides (ppt) can help you write the iterative version.
        \item \textbf{(14 points) Task \#4:} The Tree class presented in the textbook (Day 2) chapter is interesting, but it does not allow 
you to specify a new tree with a clean user interface. (a) Update the constructor (and other methods if necessary) to accomodate the creation of a tree using a Hash. 
The constructor should accept a nested structure of Hashes. You should be able to specify a tree as below. 
To test your functionality, you should traverse (by calling visit\_all method on the tree, i.e., the root node) 
this entire tree and print out the node names. (b) Write more code which counts the total number of nodes on the tree and also counts the total number of leaves on the tree.

        \begin{lstlisting}[language=Ruby]
            ruby_tree = Tree.new({
                'ggp' => { 
                    'gp1' => 
                        {'p1' => {'child1' => {}, 'child2' => {}},
                         'p2' => {'child3' => {}} 
                        }, 
                    'gp2' => 
                        {'p3' => {'child4' => {}}, 
                         'p4' => {'child5' => {}, 'child6' => {}} 
                        },
                    'gp3' => 
                        {'p5' => {'child7' => {}}, 
                         'p6' => {'child8' => {}} 
                        }
                } 
            })
        \end{lstlisting}
\emph{Writing readme carries 1 point.}

\textbf{Hint:} for part (a) you may mainly modify the \emph{initialize} function of the Tree class.
       \item \textbf{(12 points) Task \#5:} In this task, your code creates a random list of 50 shape objects, 
then traverses the list from start to end, and computes the total area of the shape objects. 
First you need to implement the class hierarchy diagram of the shape types, which is attached. 
Shape is an abstract class which has only a "color" attribute whereas (regular) 
Octagon class and (regular) Hexagon
class are concrete children of Shape class, and they have more attributes and constructors. 
Note that you do not know beforehand the order of the shape objects 
(i.e. Octagons and Hexagons) created in the random list, e.g., 
you do not know beforehand whether the 1st item is Octagon or Hexagon.
\emph{Writing readme carries 1 point.}

Note. When you traverse the list to calculate the total area, you call the area() function 
of each shape object (without considering whether it is Ocagon or Hexagon). 
That means, you will use the concept of polymorphism.

\textbf{Hint:} While building the list of shape objects, use rand(2) to generate a random number 0 or 1; if 0, then you may add a Octagon object, else add a Hexagon object to the list.
    \end{enumerate}
       
\end{document}
